\newpage\section{INTRODUCTION}\label{sec1}%Introduction

Accurate localization of sounds is required for navigation, communication, predation and escape. It is therefore not surprising that neural and structural specializations have evolved to perform this complex task. The human cortex for example continuously integrates sensory input across modalities to estimate the relative direction and distance of objects in the environment. Whereas the topography of visual and somatosensory space is represented by a point‐by‐point mapping of primary receptors, the auditory scene must first be computed from incoming sound waves. In this scenario, the direction-dependent filtering profile of the pinnae and upper body result in spectral cues that aid in the inference of sound source location (\citet{wightman_headphone_1989},  \citet{blauert_spatial_1996}). As these filters underlie lifelong changes beyond the developmental period (\citet{otte_age-related_2013}, \citet{clifton_growth_1988}), the auditory system must retain its ability to recalibrate the mapping of spectral cues to locations in space. \citet{hofman_relearning_1998} demonstrated such a recalibration experimentally by inserting molds in the concha of adult listeners. Localization accuracy was initially impaired by the modified spectral cues and recovered within a few weeks as participants adapted to the molds. Interestingly, listeners localization accuracy with their native ears remains unchanged once the molds are removed, indicating that accommodation to new spectral cues does not affect the representation of the original cues. It has thus been suggested that adapting to a new pinna shape resembles the acquisition of a new language, in that different sets of pinna filters are stored in parallel and without much interference. In this view, both learned representations are always activated by sounds and the auditory system selects the correct spectral-to-spatial mapping based on the current pinna filter. However, \citet{trapeau_fast_2016} showed that after adaptation, listeners were able to accurately localize sounds with their native ears at the first trial, without previous knowledge of the stimulus spectrum, tactile information about the absence of the molds or visual feedback. A lack of even short-lived aftereffects could be explained by a different mapping mechanism, in which several spectral profiles become associated with one spatial location and no selection between discrete maps would be required \citep{trapeau_fast_2016}. One hope in pursuing these questions is that a deeper understanding of the neural mechanisms underlying this plasticity could uncover principles which can be applied more generally. So far, studies investigating the adaptation process have focused on the effects of learning one additional set of spectral cues (see \citet{carlile_plastic_2014} for a comprehensive overview). However, simultaneous adaptation to multiple pinna shapes may reveal limitations in the auditory systems' ability to encode distinct spectral-to-spatial maps without interference and thus help to determine the underlying mechanism. In the present study participants learned to localize sounds with two different sets of earmolds within two consecutive five-day adaptation periods. To establish both representations in quick succession, participants underwent daily sessions of sensory-motor training, which has been shown to accelerate the adaptation to modified pinnae (\citet{carlile_relearning_2014}, \citet{parseihian_rapid_2012}, \citet{trapeau_fast_2016}). This paradigm was chosen to test (1) whether the second adaptation benefits from the neural plasticity induced during previous cue relearning (i.e. occurrence of metaplasticity) and (2) if the previously observed absence of aftereffects on listeners original mappings could be confirmed for a recently learned set.