Accurate localization of sounds is required for navigation, communication, predation and escape. It is therefore not surprising that neural and structural specializations have evolved to perform this complex task. The human cortex for example continuously integrates sensory input across modalities to estimate the relative direction and distance of objects (things) in the environment. Whereas the topography of visual and somatosensory space is represented by a point‐by‐point mapping of primary receptors, the auditory scene must first be computed from incoming sound waves. In this scenario, the direction-dependent filtering profile of the pinnae and upper body result in spectral cues that aid in the inference of sound source location.As these filters underlie lifelong changes, beyond the developmental period, the auditory system must maintain its ability to recalibrate the mapping of spectral cues to locations in space. [source] 
