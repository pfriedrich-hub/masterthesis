\section{Discussion}\label{sec1}%Introduction

Listeners were successively adapted to distinct sets of spectral cues for sound localization to investigate effects of metaplasticity and interference between the learned spectral mappings. Modified pinna shapes were sufficiently different to cause repeated disruption of vertical localization. Participants relearned sound localization with both sets of silicone molds within the five-day adaptation periods. Individual adaptation performance did not improve with the second set, suggesting an absence of meta-plasticity. Adaptation persistence did not differ between earmolds, showing that learning a second set of modified spectral cues does not degrade the previously learned representation. In contrast to previous findings, adaptation to new spectral cues reduced participants vertical accuracy with free ears.

\subsection{Quantification of spectral features differences between earmolds}

Application of silicone molds modified participants DTFs in a wide range of frequencies (figure) and resulted in large behavioral effects. To quantify behaviorally relevant acoustic changes induced by the consecutive pinna modifications, the frequencies of salient spectral features were initially identified. Previous findings show that spectral cues which may guide vertical sound localization are not equally distributed across frequencies. Instead, distinct features such as notches and peaks that vary with elevation tend to be situated in specific frequency bands (sources). In the cat, these features are extracted by edge detecting neurons in the dorsal cochlear nucleus which are thought to be associated with spectral cue processing (Davis et al. 2003). Two previously introduced measures, VSI (source) and spectral strength (source), were used to compare the amount of spectral information for elevation discrimination in individuals free ears across five octave bands. Both measures varied significantly between frequency bands and were highest in the 5.7 – 11.3 kHz (VSI) and 6.7 – 13.5 kHz (spectral strength) band, reflecting the contribution of prominent acoustic features located in these bands; a spectral notch and its neighbouring peak between 5 and 14 kHz (see figure). 

VSI reduction due to molds 

(sources) reported a systematic variation in the frequencies of spectral features among individuals, which is indicated by a relatively broad peak and large error bars of VSI across frequency bands in (figure). In the 5.7 – 13.5 kHz band, VSI was positively correlated with individual localization accuracy on the vertical axis, suggesting that spectral cues in this band contributed to the estimation of sound elevation. 



\newpage
 peak sensitive neurons?
 left right
 
To assess the 
he directional filter functions of participants ears 
the behaviroal relevance of the spectral cues

efficacy of spectral cues for vertical localization varies across frequency bands. 

The efficacy of spectral cues 
To establish a link between the acoustic properties of the pinna and vertical localization performance 

behavioral impact of the earmolds, spectral cues were analysed in frequency bands between 4 and 16 kHz 
to what extent these differences were explained by acoustic differences in directional transfer functions caused by different ear shapes

Spectral features guiding vertical sound localization 

Spectral changes occured at all frequency bands (figure)
but prominent features are located at specific frequencies (figure spectral)
trapeau has shown that these features are behaviorally relevant
middlebrooks has shown that features vary among individuals
different measures were used to locate spectral information and quantify behaviroal relevance of that band 



To 

To asses the magnitude of mold induced differences in spectral information that were relevant for vertical localization, previously introduced measures (VSI source, spectral strength source) 


To compare acoustic differences between individuals free and both sets of modified ears that were behaviorally relevant

 of the To compare earmolds

Prominent spectral features were alleviated 

Peaks and notches frequency bands

To compare acoustic properties of modified pinna shapes with respect 

earmolds in their acoustic 

acoustic changes across molds and participants, 

the magnitude of acoustic changes with respect to their behavioral relevance was previously introduced spectral measures were first tested 

measure acoustic changes 

To determine whether 

acoustic properties  pinna acoustics and subsequently reduced vertical localization performance.


To establish the link between the acoustic properties of individual pinna shapes and vertical localization performance, previously introduced measures of spectral information for vertical sound localization were compared with behavioral performance. 

VSI and spectral strength 


resulted in strong behavioral effects, similar to previous studies. 

- show that vsi is valid and test behavioral relevant bandwidth

To assess acoustic differences induced by pinna modifications and their behavioral relevance, VSI and spectral strength were measured across different bands previously introduced spectral measures VSI and spectral strength measures were first compared to 

behvaioral 

Localization performance with free ears differed among individuals. 

As previously shown Acoustic factors could explain some of that variation (VSI trapeau 2016 bandwidth 5.7-11.3) 
middlebrooks showed that spectral cues varied among individuals within larger band 
spectral strength = saliency
vsi = discriminability
saliency indicated spectral features extend to higher bands (spectral notches play a role (sources))
this is where spectral features explained localization performance with free ears

spectral peaks (higher bands) also important (figures, sources) and affected by molds

this bandwidth was subsequently chosen to analyse acoustic effects of earmolds






\subsection{Behavioral effects}

\subsection{Adaptation}

\subsection{Aftereffect}

\subsection{Persistence}




acoustics
A. VSI and localization performance with free ears were related
B. Earmolds reduced spectral information in the 5.7-11.3 kHz band
C. Acoustic changes occurred at similar frequencies across participants and were dif- ferent between consecutive molds
D. Consecutive pinna modifications were equally dissimilar and physiologically plau- sible
behavior
E Earmolds reduced vertical localization performance
F. Participants adapted to two novel sets of DTFs
G. No indication of metaplasticity in the adaptation to the second set of earmolds
I. RMSE with free ears was increased after mold removal
J. Vertical localization accuracy with the first earmolds decreased during adaptation to the second molds


second detailled discussion of results

different learned sensory representations.

brain mechanisms underlying adaptation to alterations in sensory input. 


Adaptation acoustic and behavior differences m1 m2 --> 

Persistence --> interference of learning m2 on m1 persistence

Aftereffect --> raised rmse throughout the experiment



A. Acoustic factors of behavioral performance
B. Effect of the molds on horizontal sound localization
C. Individual differences in adaptation to modified spectral cues
D. Aftereffect
E. Persistence

