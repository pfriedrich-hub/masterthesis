\section{Discussion}\label{sec1}%Introduction

Listeners were successively adapted to distinct sets of spectral cues for sound localization to investigate effects of metaplasticity and interference between the learned spectral mappings. Modified pinna shapes were sufficiently different to cause repeated disruption of vertical localization. Participants relearned sound localization with both sets of silicone molds within successive five-day adaptation periods. Individual adaptation performance did not improve with the second set, suggesting an absence of meta-plasticity. Adaptation persistence did not differ between earmolds, showing that learning a second set of modified spectral cues does not degrade the previously learned representation. In contrast to previous findings, adaptation to new spectral cues reduced participants vertical accuracy with free ears.

\subsection{Acoustic comparison of the successive earmolds}
%\subsection{Acoustic effects of pinna modifications}

Application of silicone molds modified participants DTFs in a wide range of frequencies (see \cref{fig:spectral_change}) and resulted in large behavioral effects. This study aimed to compare the effects of learning multiple sets of spectral cues on adaptation rate and persistence. To enable this comparison, the magnitude of acoustic changes induced by the two sets of earmolds should be similar. It has been shown that larger acoustic differences between free and modified ears caused a stronger decrease in vertical localization performance \citep{wanrooij_relearning_2005} and reduced the rate of subsequent adaptation to the new spectral cues \citep{trapeau_fast_2016}. To quantify behaviorally relevant changes induced by the consecutive pinna modifications, the frequencies of salient spectral features were initially identified. Previous findings show that spectral cues which may guide vertical sound localization are not equally distributed across frequencies. Instead, distinct features such as notches and peaks that vary with elevation tend to be situated in specific frequency bands (\citet{langendijk_contribution_2002}, \citet{trapeau_fast_2016}). In the cat, these features are extracted by edge detecting neurons of the DCN which are thought to be associated with spectral cue processing (\citet{davis_auditory_2003}, \citet{reiss_spectral_2005}). Two previously introduced measures, VSI and spectral strength, were used to compare the amount of spectral information in participants' free ears across frequency bands. Both measures varied significantly between the five octave bands and were highest in the 5.7–11.3 kHz (VSI) and 6.7–13.5 kHz band (spectral strength), reflecting the contribution of prominent acoustic features located in these bands (the spectral notch and its neighbouring peak between 5 and 14 kHz in figure \cref{fig:spectral_change} A). \citet{middlebrooks_individual_1999} reported a systematic variation in the frequencies of spectral features among individuals, which is indicated by the broad peak and large error bars of VSI across frequency bands in \cref{fig:ef_vsi} A. VSI between 5.7 and 13.5 kHz varied between individuals and was positively correlated with individual localization accuracy on the vertical axis, suggesting that acoustic features in this band largely contributed to the estimation of sound elevation. In accordance with previous studies, silicone molds attenuated the spectral notch between 5.7 and 11.3 kHz (\citet{hofman_relearning_1998}, \citet{wanrooij_relearning_2005}, \citet{trapeau_fast_2016}), indicated by a reduction of VSI in this band after mold insertion. Importantly, this reduction was of similar magnitude for the first and the second set of earmolds (see \cref{fig:molds_vsi} B). The volumes of individual silicone molds were adjusted to participants concha volumes aiming to induce consistent disruption of elevation perception (EG) across individuals. Compared to the first earmolds, the second set required noticeably larger volumes to achieve similar levels of EG reduction, indicated by the increased VSI dissimilarity between participants' unmodified ears and the second molds in \cref{fig:molds_vsi} D. Despite these variations in volume, each set induced acoustic changes at similar frequencies across individuals (see \cref{fig:spectral_change} D, E). Intuitively, smaller concha volumes interact with shorter wavelengths, which is reflected by an upward shift in the frequencies of marked changes induced by the second set. This effect might also result in the formation of new spectral cues at higher frequencies \citep{wanrooij_relearning_2005}, and could be shown for the first set of earmolds, which increased VSI between 11.3 and 13.5 kHz compared to unmodified ears. The experiment was designed to answer two questions: (1) whether learning a set of spectral cues facilitates subsequent adaptation to a second set and (2), if learning a second set interferes with a previously learned representation. VSI dissimilarity was negatively related to adaptation performance in a previous study using modified pinnae \citep{trapeau_fast_2016}, and was therefore taken into account when interpreting behavioral results. In this view, acoustic similarities between both sets of molds could facilitate adaptation to the second set, because little cue reweighing would be required to accommodate to the new spectral mapping. Moreover, learning a second set of DTFs with similar spectral cues could enhance the persistence of the previously learned spectral mapping and counter possible effects of interference due to the repeated cue relearning. However, VSI dissimilarities did not differ between successive pinna shapes (\cref{fig:molds_vsi} D), suggesting that behavioral results are unlikely to be confounded by disproportionate acoustic differences. 

\subsection{Behavior}

Earmolds reduced vertical performance  

\subsection{Metaplasticity and Interference}
\subsection{rate and persistence }




\newpage

adaptation ranged from minor to full recovery  

Large individual differences in adapta- tion to modified localization cues were previously reported (spectral cues: Carlile and Blackman, 2013; Hofman et al., 1998; Van Wanrooij and Van Opstal, 2005; ITDs: Javer and Schwarz, 1995; Trapeau and Sch€onwiesner, 2015; supernor- mal cues: Shinn-Cunningham et al., 1998).




behavior
E Earmolds reduced vertical localization performance
F. Participants adapted to two novel sets of DTFs
G. No indication of metaplasticity in the adaptation to the second set of earmolds
I. RMSE with free ears was increased after mold removal
J. Vertical localization accuracy with the first earmolds decreased during adaptation to the second molds


A. Acoustic factors of behavioral performance
B. Effect of the molds on horizontal sound localization
C. Individual differences in adaptation to modified spectral cues
D. Aftereffect
E. Persistence

acoustics
A. VSI and localization performance with free ears were related
B. Earmolds reduced spectral information in the 5.7-11.3 kHz band
C. Acoustic changes occurred at similar frequencies across participants and were dif- ferent between consecutive molds
D. Consecutive pinna modifications were equally dissimilar and physiologically plau- sible


\subsection{Behavioral effects}

\subsection{Adaptation}

\subsection{Aftereffect}

\subsection{Persistence}

second detailled discussion of results

different learned sensory representations.

brain mechanisms underlying adaptation to alterations in sensory input. 

Adaptation acoustic and behavior differences m1 m2 --> 

Persistence --> interference of learning m2 on m1 persistence

Aftereffect --> raised rmse throughout the experiment




