\section{Discussion}\label{sec1}%Introduction

Listeners were successively adapted to distinct sets of spectral cues for sound localization to investigate effects of metaplasticity and interference between the learned spectral mappings. Modified pinna shapes were sufficiently different to cause repeated disruption of vertical localization. Participants relearned sound localization with both sets of silicone molds within consecutive five-day adaptation periods. Individual adaptation performance did not improve with the second set, suggesting an absence of meta-plasticity. Adaptation persistence did not differ between earmolds, showing that learning a second set of modified spectral cues does not degrade the previously learned representation. In contrast to previous findings, adaptation to new spectral cues reduced participants vertical accuracy with free ears.

%\subsection{Acoustic comparison of the successive earmolds}
\subsection{Acoustic comparison of pinna modifications}

Application of silicone molds modified participants DTFs in a wide range of frequencies (see \cref{fig:spectral_change}) and resulted in large behavioral effects. This study aimed to compare the effects of learning multiple sets of spectral cues on adaptation rate and persistence. To enable this comparison, the magnitude of acoustic changes caused by the two sets of earmolds should be similar. It has been shown that larger acoustic differences between free and modified ears caused a stronger decrease in vertical localization performance \citep{wanrooij_relearning_2005} and reduced the rate of subsequent adaptation to the new spectral cues \citep{trapeau_fast_2016}. To quantify behaviorally relevant changes induced by the consecutive pinna modifications, the frequencies of salient spectral features were initially identified. Previous findings show that spectral cues which may guide vertical sound localization are not equally distributed across frequencies. Instead, distinct features such as notches and peaks that vary with elevation tend to be situated in specific frequency bands (\citet{langendijk_contribution_2002}, \citet{trapeau_fast_2016}). In the cat, these features are extracted by edge detecting neurons of the DCN which are thought to be associated with spectral cue processing (\citet{davis_auditory_2003}, \citet{reiss_spectral_2005}). Two previously introduced measures, VSI and spectral strength, were used to compare the amount of spectral information in participants' free ears across frequency bands. Both measures varied significantly between the five octave bands and were highest in the 5.7–11.3 kHz (VSI) and 6.7–13.5 kHz band (spectral strength), reflecting the contribution of prominent acoustic features located in these bands (the spectral notch and its neighbouring peak between 5 and 14 kHz in figure \cref{fig:spectral_change} A). \citet{middlebrooks_individual_1999} reported a systematic variation in the frequencies of spectral features among individuals, which is indicated by the broad peak and large error bars of VSI across frequency bands in \cref{fig:ef_vsi} A. VSI between 5.7 and 13.5 kHz varied between individuals and was positively correlated with individual localization accuracy on the vertical axis, suggesting that acoustic features in this band largely contributed to the estimation of sound elevation. In accordance with previous studies, silicone molds attenuated the spectral notch between 5.7 and 11.3 kHz (\citet{hofman_relearning_1998}, \citet{wanrooij_relearning_2005}, \citet{trapeau_fast_2016}), indicated by a reduction of VSI in this band after mold insertion. This reduction did not differ significantly between the first and the second set of earmolds (see \cref{fig:molds_vsi} B). The volumes of individual silicone molds were adjusted to participants concha volumes aiming to induce consistent disruption of elevation perception (EG) across individuals. Compared to the first earmolds, the second set required noticeably larger volumes to achieve similar levels of EG reduction, indicated by the increased VSI dissimilarity between participants' unmodified ears and the second molds in \cref{fig:molds_vsi} D. Despite these variations in volume, each set caused acoustic changes at similar frequencies across individuals (see \cref{fig:spectral_change} D, E). Intuitively, smaller concha volumes interact with shorter wavelengths, which is reflected by an upward shift in the frequencies of marked changes induced by the second set. This effect might also result in the formation of new spectral cues at higher frequencies \citep{wanrooij_relearning_2005}, and could be shown for the first set of earmolds, which increased VSI between 11.3 and 13.5 kHz compared to unmodified ears. The experiment was designed to answer two questions: (1) whether learning a set of spectral cues facilitates subsequent adaptation to a second set and (2), if learning a second set interferes with a previously learned representation. VSI dissimilarity was negatively related to adaptation performance in a previous study using modified pinnae \citep{trapeau_fast_2016}, and was therefore taken into account when interpreting behavioral results. In this view, acoustic similarities between both sets of molds could facilitate adaptation to the second set, because little cue reweighing would be required to accommodate to the new spectral mapping. Moreover, learning a second set of DTFs with very similar spectral cues could enhance the persistence of the previously learned spectral mapping and counter possible effects of interference due to the repeated cue relearning. However, VSI dissimilarities did not differ between successive pinna shapes (\cref{fig:molds_vsi} D), suggesting that hypothesis testing is unlikely to be confounded by disproportionate acoustic differences. 

\subsection{Adaptation to successive earmolds}

Insertion of earmolds significantly reduced vertical localization performance. This effect varied across individuals and between successive molds and was not fully explained by acoustic dissimilarities between pinna shapes (\cref{fig:vsi_dis_rmse}). Previous studies suggested that individual performance might also depend on non-acoustic factors such as attention, perceptual abilities and neural processes underlying spectral feature analysis. Differences in initial localization with the first and second molds were more pronounced for sounds located at lower elevations (\cref{fig:response_evo}). Apart from the outer ear, the head and torso also interact with incoming sound waves and provide coarse elevation cues for sources below the horizontal midline. Spectral cues created by the torso are situated at frequencies below 3 kHz (\citet{asano_role_1990}, \citet{algazi_elevation_2001}), and were unlikely to be affected by the earmolds. Participants might have learned to partially rely on these cues during the first adaptation, causing the observed differences in performance reduction between both sets. Localization performance improved to various degrees throughout the five day adaptation periods. Individual differences varied continuously from minor adaptation to full recovery. Previous studies reported strong inter-individual variation in adaptation to spectral cues (\citet{hofman_relearning_1998}, \citet{wanrooij_relearning_2005}, \citet{trapeau_fast_2016}). Elevation perception recovered more quickly with the first set of earmolds than with the second set, likely due to the larger acoustic changes these molds induced to participants native ears. Adaptation rate did not increase with repeated cue relearning within the time span of this experiment. Longer periods of cue relearning might be necessary to induce the phsyiological changes associated with metaplasticity. Earlier studies found no aftereffects on localization with free ears when the adapted earmolds were removed (\citet{carlile_relearning_2014}, \citet{hofman_relearning_1998}, \citet{trapeau_fast_2016}, \citet{wanrooij_relearning_2005}), indicating that learned spectral-to-spatial mappings do not override pre-existing ones. In agreement with these results, adaptation persistence did not differ significantly between the successive earmolds, suggesting that adaptation to the second set of molds did not interfere with the previously learned mapping. Vertical accuracy (RMSE) with the first molds was decreased after learning a second set of spectral cues, and a similar decrease in accuracy was observed for individuals' performance with free ears. However, free ears accuracy simultaneously decreased on the horizontal plane. A loss of general localization accuracy throughout the experiment could be explained by procedural factors such as participants trying to quickly complete the localization task. A model proposed by \citet{hofman_spectro-temporal_1998} posits that sound source elevation is estimated by comparing the spectrum of incoming sounds with spectral templates associated to different locations. After the second adaptation period, participants were able to access three different sets of spectral cues for sound localization, which required the underlying neural populations to accommodate two additional spectral mappings in quick succession and with minimal interference. This raises the question, at which stage of the auditory pathway the learned spectral templates are represented. The required neural plasticity might arise in subcortical structures. DCN neurons and their projections to the central nucleus of the ICC have been shown to encode spectral features in the cat \citep{davis_auditory_2003} and evidence exists for a topographical representation of auditory space in the SCC of the ferret \citep{king_spatial_1987}. However, the ability to learn multiple sets of spectral cues without interference indicate that spectral maps might be represented at the level of the auditory cortex. Recent studies found tuning curves in low-level AC encoding sound elevation, which were flattened after pinna modification and recovered their original shape as participants adapted to modified cues and regained elevation perception \citep{trapeau_encoding_2018}. To better understand the 
Future research could test limitations  outlook - more control over DTF manipulation, VR, 3d printed molds ...
Is there a limited number? 

